\chapter{readme}
\label{md__c_1_2prj_2python_2_f_semylator_2task2_2git_2readme}\index{readme@{readme}}

\begin{DoxyEnumerate}
\item Общее описание. Разработать инструмент командной строки для визуализации графа зависимостей, включая транзитивные зависимости. Сторонние средства для получения зависимостей использовать нельзя. 80 Зависимости определяются для git-\/репозитория. Для описания графа зависимостей используется представление Graphviz. Визуализатор должен выводить результат на экран в виде кода. Построить граф зависимостей для коммитов, в узлах которого содержатся дата, время и автор коммита. Конфигурационный файл имеет формат toml и содержит\+: • Путь к программе для визуализации графов. • Путь к анализируемому репозиторию. • Путь к файлу-\/результату в виде кода. Все функции визуализатора зависимостей должны быть покрыты тестами.
\item Описание всех функций и настроек. Описание функций load\+\_\+config(config\+\_\+file\+: str) -\/$>$ dict
\end{DoxyEnumerate}

Загружает настройки конфигурации из указанного INI-\/файла. Аргументы\+: config\+\_\+file\+: Путь к INI-\/файлу конфигурации. Возвращает\+: Словарь, содержащий настройки конфигурации. Поведение\+: Читает INI-\/файл и парсит его секции и параметры. get\+\_\+commits(repo\+\_\+path\+: str, since\+\_\+date\+: str) -\/$>$ List[Tuple[str, str]]

Извлекает список коммитов из указанного Git-\/репозитория с заданной даты. Аргументы\+: repo\+\_\+path\+: Путь к Git-\/репозиторию. since\+\_\+date\+: Строка даты (например, "{}2023-\/01-\/01"{}), начиная с которой нужно получить историю коммитов. Возвращает\+: Список кортежей. Каждый кортеж содержит\+: Хэш коммита. Дату коммита в формате "{}\+YYYY-\/\+MM-\/\+DD HH\+:\+MM "{}. Автора коммита. Исключения\+: Вызывает исключение, если выполнение команды Git завершается с ошибкой. build\+\_\+dependency\+\_\+graph(commits\+: List[\+Tuple[str, str]]) -\/$>$ Digraph

Создает граф зависимостей коммитов с использованием Graphviz. Аргументы\+: commits\+: Список данных о коммитах в хронологическом порядке (хэш коммита, дата, автор). Возвращает\+: Объект Digraph, представляющий граф зависимостей. Поведение\+: Создает узел для каждого коммита и соединяет их в хронологическом порядке. save\+\_\+graph(graph\+: Digraph, output\+\_\+file\+: str) -\/$>$ None

Сохраняет сгенерированный граф зависимостей в файл формата PNG. Аргументы\+: graph\+: Объект Digraph, который нужно сохранить. output\+\_\+file\+: Путь к файлу для сохранения (без расширения). Поведение\+: Сохраняет граф в формате PNG по указанному пути. main(config\+\_\+file\+: str) -\/$>$ None

Главная функция, которая организует загрузку конфигурации, извлечение коммитов, создание графа и его сохранение. Аргументы\+: config\+\_\+file\+: Путь к INI-\/файлу конфигурации. Поведение\+: Загружает настройки, получает данные о коммитах, строит граф зависимостей и сохраняет его. Настройки конфигурационного файла Конфигурационный файл (config.\+ini) должен содержать следующие секции и ключи\+:

[Settings] repository\+\_\+path=C\+:/prj/python/\+FSemylator/.git graph\+\_\+output\+\_\+path=C\+:/prj/python/\+FSemylator/task2/git/out since\+\_\+date=2024-\/11-\/01 graphviz=C\+:\textbackslash{}\+Program Files\textbackslash{}\+Graphviz\textbackslash{}bin


\begin{DoxyEnumerate}
\item Описание команд для сборки проекта. python show\+\_\+commits pip install graphviz и установить саму программу Graphviz
\item Примеры использования в виде скриншотов, желательно в анимированном/видео формате, доступном для web-\/просмотра. 
\item Результаты прогона тестов. Testing started at 19\+:15 ... Launching unittests with arguments python -\/m unittest C\+:\textbackslash{}prj\textbackslash{}python\textbackslash{}\+FSemylator\textbackslash{}task2\textbackslash{}git\textbackslash{}test.py in C\+:\textbackslash{}prj\textbackslash{}python\textbackslash{}\+FSemylator\textbackslash{}task2\textbackslash{}git
\end{DoxyEnumerate}

result\+: abc123 1672531200 Anika xyz789 1672617600 Anna

Ran 4 tests in 0.\+010s 